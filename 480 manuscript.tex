\documentclass[12pt]{article}
\usepackage{fullpage}
\usepackage{biblatex} [style=Chicago]
\usepackage{indentfirst}
\usepackage{caption}

\makeatletter
\renewcommand\@biblabel[1]{}
\makeatother

\begin{document}
\title{Examining the Undecided Voter's Response to Presidential Debate Issue Statements}
\author{Lauren Palladino}
\date{Rice University}
\maketitle

\section*{Abstract}

Televised presidential debates have been a staple of campaigning and political culture in the United States ever since 1960. While extensive research exists on the effect of debates on the general American electorate, this project specifically focuses on the effects of debates on the undecided voter. Specifically, this research focuses on the effects of issue statements presented in debates, rather than visual components such as poise, demeanor, and appearance. Data was collected from a survey yielding 206 individual responses from a diverse array of backgrounds. A hypothetical election was established, and subjects read a series of ten issue statements made in this hypothetical election. After considering statements, subjects indicate their preference, or lack thereof, for one of the candidates. Though initial analysis led to insignificant results in the interaction between voters without a strong party identification and an undecided vote in this hypothetical election, follow up analyses indicated significant interactions in both race and physical ability in the inability to cast preference for our hypothetical candidates. \\

\pagebreak

The first televised presidential debate in 1960 between Nixon and Kennedy prompted some of the first scholarly literature on the effectiveness and purpose of debates on American voters. Initial research broadly found that the 1960 debate proved to be a significant force in shaping vote choice in those who watched (Middleton, 1962). In this work, I will instead be focusing on individuals who do not identify with a major party, and their experiences with presidential debates. It is widely accepted and understood that presidential debates enforce existing partisan divides amongst decided voters (Jarman, 2005; Holbert, 2005). Furthermore, there seems to be a general notion that undecided voters with weak partisan attachments are more likely to be influenced by debates (Geer, 1988; Hillygus and Jackman, 2003). Debates are designed to educate and sway the public. This research will  answer whether they truly sway the most variable members of the voting population: the undecided voter.\\

As candidates attempt to use televised debates to sway their audience, it’s clear that undecided voters play a key role. Even once undecided voters were invited to debates to ask questions of candidates, researchers have found that the clarity and substance of their questions have been perfectly comparable to traditional media moderators (Eveland, McLeod, and Nathanson, 1994). Furthermore, the undecided voter population tends to hold a different issue agenda from the politically educated subset (Benoit and Hansen, 2001). There exists not only a lack of research, but a lack of understanding regarding the role of the undecided voter in presidential debates. This research sets out to understand how this subset of the voting public responds to these events, ultimately concluding whether debates are accomplishing their intended goal of swaying the public.\\


\section*{Weighing Content and Visual Components of Presidential Debates}

Historically, debate candidates choose to devote more time on attacks, rather than developing issue platforms (Benoit and Harthcock, 1999). Thus, when viewing debates, voters have to weigh many factors including candidate poise, demeanor, rhetoric, and issue statements themselves. While there is a wealth of research on visual components of debates, my work will only be focusing on the policy platforms and issue statements in debates. Though presented with a range of information, viewers tend to weigh policy positions more heavily than other factors (Sigelman and Sigelman, 1984). Candidates go to great lengths to build policy platforms and frame their issues strategically in televised debates in hopes of swaying the audience to their side. The visual components of debates such as demeanor and rhetoric are often strategically used to frame issue statements (Jerit, 2008). What happens when these issue statements are considered in isolation? In their 1984 publication, Sigelman and Sigelman seek to explain how voters determine who ‘wins’ each presidential debate, and ultimately acknowledge that the ‘winner’ can almost always be determined by the extent to which viewers agree with issue statements. \\

Without firmly decided views, individuals without attachment to either major party tend to make their political choices based on their implicit attitudes and emotions (Christ, 1985; Arcuri, Castelli, Galdi, Zogmaister, and Amadori, 2008). This decision making process will be facilitated through a hypothetical election. There also exists an issue in how researchers define the undecided population. Many choose to include individuals with weak partisan attachments to major parties as undecided voters (Fenwick, Wiseman, Becker, and Heiman, 1982). Likewise, this research will focus in individuals who do not identify with a major party.\\


\section*{Undecided Voters as Uninformed Voters}

Beginning in the 1960s, the United States saw large dealignment, resulting more voters refusing to identify with either major party (Clarke and Suzuki, 1994). With this increase in independent voters in the electorate, many candidates see this as an opportunity to appeal to a broad audience as undecided voters can sway entire elections (Großer and Schram, 2010; Wegenast, 2010). Candidates took the opportunity to sway a subset of the electorate who may not have preexisting, firm views on issues in the election. While undecided voters hold large electoral power, their behavior is difficult to predict. Opinion polls are able to quite accurately predict the partisan vote, but the large population of undecided voters are largely unpredictable (Gelman and King, 1993). \\

It can be theorized in the context of this research that undecided voters are largely uninformed, so when asked to consider if quotations align with their beliefs, they will struggle and come to inconsistent results. In allocating campaign funding, candidates have historically attempted to reach out to the uninformed voter (Baron, 1994). In these circumstances, politicians attempt to educate undecided and uninformed and move them to their side. Examining the historical attempts of politicians to gain the undecided and uninformed vote, we can understand the inherent value of these voters to the political campaigning process. Once we consider the interconnected nature of undecided and uninformed voters, we can theorize that individuals who do not identify with a major party will be unable to state that one candidate is preferable to the other, and remain undecided.\\


\section*{Issue Statements as a Means of Ideological Alignment}

Existing models used to explain the effectiveness of debates focus on two primary theories: agenda setting theory and democratic theory (McKinney and Carlin, 2004). While debates are able to connect to voters and implement a sense of policy priorities, my research will use a model to focus more specifically on the behavior of undecided voters. Candidates in debates make both issue and non-issue related statements, the issue and policy statements are often neglected by voters, particularly uninformed voters (Abrajano, 2005).  By focusing exclusively on issue statements, my work will emphasize the role that policy plays in debates. While individuals who watch debates may not always be able to pick one candidate over the other as the ‘winner,’ debates will certainly increase overall issue knowledge (Holbrook, 1999; Benoit, Hansen, and Verser, 2003). By developing the existing research on debate policy statements, it is my hope that a more clear interaction between voters and policy positions may be found. Because candidates use debates as a platform to advocate for policy, it is imperative that the success of these attempts be researched, and measured.\\

While issue statements themselves may not sway voters, the way in which candidates deliver their debate messages has large potential to influence voters (Clementson, Pascual‐Ferrá, and Beatty, 2016). With a wealth of existing literature in presidential demeanor, tone, and delivery, this work will refocus debates on the strict substance of debate statements. In previous research using hypothetical elections, voters have struggled to align themselves with hypothetical issue statements (Williams, 1994). Generally speaking, voters struggle to identify issues central to political candidates’ campaigns, (Dalager, 1996) and this phenomenon will only be magnified when focusing on the undecided voter.\\


\section*{Hypothesis}

Individuals who do not identify with a major party will not be able to accurately align themselves with the debate candidates. Thus we can expect that the voters that did not voice preference for a major party will not be swayed towards either candidate in the hypothetical election.\\

\section*{Methods}

Traditionally when studying presidential debates, scholars have employed a series of surveys before and after debates (Blais and Perrella, 2008). In this project, I will be implementing a series of surveys where participants will be presented with  a series of policy statements made in debates. In doing so, participants in this study will be focusing their vote choice on the issue statements rather than the visual components of the debate. This distinction is a key component of my research, as both objective issue statements and subjective visual perception can lead to large differences in voter opinion (Lanous and Schrott, 1989). While the method of before-and-after surveys is nothing new, my work employs a means of narrowing the focus of the study. Overall, my method is integral to establishing correlation between vote choice, and the issue statements made in debates.\\

After gathering demographic information such as ethnicity, gender, and highest level of education completed, subjects in this study are asked to indicate their party identification. Subjects were asked if they considered themselves a Republican, Democrat, Independent, or what? Individuals who selected other were asked to manually enter their identification. Those that responded as Republican or Democrat were later asked to identify whether they were a strong or not strong identifier of that party. Subjects who identified as an Independent were subsequently asked if they leaned closer to the Republican party, the Democratic party, or neither party. Following this section, subjects are presented with the instructions:\\


\noindent{“Candidate A and candidate B are running for president. Please read some of their issue statements below.”\\

A series of issue statements are subsequently listed. For example, consider the following statements on energy policy:\\

\noindent{Candidate A: “We have to guard our energy companies. The EPA are [sic] so restrictive that they are putting our energy companies out of business” \\
\noindent{Candidate B: “We are for the first time energy independent. We do not rely on the Middle East… I think we can be the 21st century clean energy superpower” \\

Four more sets of statements follow on issues involving economic policy, the United States’ role in Iraq, healthcare, and supreme court nominations. After considering the five sets of issue statements, subjects are finally asked:\\

\noindent{“Considering these statements, are you moved to vote for either of the two candidates? If so who?” Subjects are given three choices: “I am undecided / have no preference,” “Candidate A,” and “Candidate B.”\\

All ten statements presented to survey participants have been taken from transcripts of various presidential debates between Donald Trump and Hillary Clinton in 2016. In each set of statements, Trump’s statements are represented by candidate A, while Clinton’s are candidate B. Once data was collected, the subjects who were unable to voice a preference for neither candidate A nor candidate B will be examined, with specific interest in whether subjects without party preferences were able to come to such a decision. \\

Given constraints such as time restrictions and lack of financial resources, it was difficult to create an ideal representation of a presidential debate. In ideal circumstances, all subjects who were free of preconceived political inclinations would be able to watch entire debates and cast their hypothetical vote afterwards. This design was simply not possible. Thus, the design was altered to only focus on issue statements. By using real statements made in real presidential debates, this experimental design replicated the conditions in the closest way possible. Even in the absence of watching the actual debate, subjects experienced the exact same issue statements that would have been made in a televised version of the debate. \\

Potential for methodological error comes from the fact that we cannot see what subjects would do in all possible scenarios involving presidential debates. To avoid this issue, all survey participants were given the exact same treatment, placing all individuals on equal footing. Specifically, subjects will be focusing on the political rhetoric of the debates so any personal biases and judgements can be avoided. Asking questions on the policy statements of the candidates will manually refocus the attention of the participants to the candidates policy positions.\\


\section*{Data}

Data were collected using an internet survey hosted by Qualtrics from March 27-April 1, 2019. Respondents were recruited using Amazon's Mechanical Turk (MTurk) service. Although MTurk is not nationally representative, studies have found that MTurk workers are valid source of data for research on political ideology and other related concepts (Clifford et al. 2015). In this study, the dependent variable was the undecided voter. To extract the undecided voter from the population of survey data, all subjects who did not identify as a Republican or Democrat were isolated into a separate subset of individuals. The independent variable examined is the issue statement questionnaire. At the end of the survey, individuals were asked to voice their preference for either candidate A, candidate B, or state that they were undecided / had no preference. \\

\begin{table}[h]
\caption{Summary Data}
\centering
 \begin{tabular}{||c c c||}
 \hline
 Factor & Mean & n \\ [0.5ex]
 \hline\hline
 Female & 0.393 & 81 \\
 Non-partisan  & 0.257 & 53 \\
 Strong partisan  & 0.570 & 86 \\
 White  & 0.689 & 142 \\
 Black & 0.131 & 27 \\
 Native American / Alaskan & 0.024 & 5 \\
 Hispanic / Latinx  & 0.068 & 14 \\
 Asian & 0.112 & 23 \\
 Other race & 0.010 & 2 \\
 Disabled  & 0.010 & 20 \\ 
 Total & & 206 \\[1ex]
 \hline
 \end{tabular}
 \end{table}


\begin{table}[h]
\caption{Non-partisan Subjects' Inability to decide on a Hypothetical Candidate}
\centering
 \begin{tabular}{||c c c||}
 \hline
 Factor & t & p \\ [0.5ex]
 \hline\hline
 Non-partisan & -0.565 & 0.578 \\ [1ex]
 \hline
 \end{tabular}

 \end{table}
 
To analyze the interaction between these two factors a two-tailed t test was conducted to examine whether the undecided voter remained undecided significantly more than they chose a candidate. Ultimately it was found that undecided voters did not remain undecided in the hypothetical election significantly more than undecided voters who cast preference for one of the two hypothetical candidates (t=-0.575, df = 21, p=0.578).\\

\section*{Undecided Voters do not Remain Unswayed}

Examining the table above, we see that individuals who did not identify with a major party did not remain undecided after the hypothetical election significantly more than they chose one of the two candidates. There was no significant difference in undecided voters remaining undecided after the hypothetical election and undecided voters indicating a candidate preference (t=-0.575, df = 21, p=0.578).\\

Once conducting a thorough investigation into the relationship between undecided voters and presidential debates, counterintuitive results were reached. While it was hypothesized that undecided voters would be unable to voice preference for one candidate over the other, we found that the opposite was true. Examining the t value in the above table (t=-0.575), we can conclude that, though not significant, individuals who did not identify with a major party actually voiced preference for one candidate more than they remained undecided in the hypothetical election. \\

After examining these results through the lens of our model, that undecided voters are effectively uninformed voters, it becomes clear that there are more complex forces at work. Because these results were not as anticipated, further analysis and considerations must be taken. It may be impossible to derive conclusions from the data collected in this survey, but it is evident that individuals who did not voice preference for a major party did not remain significantly undecided in this election. \\

\section*{Conclusion}

After examining the existing literature on undecided voters, the history of debate research, and implementing a hypothetical election to a set of 206 subjects, we found that there was no significant difference in individuals without a major party identification remaining undecided after the hypothetical election and those who indicated a candidate preference. Despite reaching counterintuitive results, examining the interactions between voters who do not identify with a major party and candidate preference provides insight into the validity of the model. This study was modeled under the pretense that undecided voters are largely uninformed, which would make choosing between two hypothetical candidates difficult. This model did not support the initial hypothesis in this research, which begs further insight into the forces at play in determining the behavior of undecided voters. \\

Initial observations suggest that contrary to the initial hypothesis, undecided voters were able to more easily reach a decision as to which hypothetical candidate they preferred. Perhaps these findings were due to the lack of information undecided voters had about the election going in. The model, that undecided voters are uninformed, has potential to work opposite of what was initially hypothesized. It is quite possible that because undecided voters are uninformed, they were more susceptible to partisan influence. \\

While the issue statements in the hypothetical election were not attached to any party labels, undecided voters may have been able to attach the statements to personal values rather than political. Individuals who initially identified as Democrats or Republicans were likely more cognizant of partisan lines, and were careful to cast their preference for the hypothetical candidate that matched their views. Unlike the subjects who identified with a major party, the non-partisan subjects did not have to exercise caution when matching their identification, or lack thereof, with a candidate. The undecided voters cast their preference exclusively on personal information and values. It is possible that the lack of party identification made casting preference for one hypothetical candidate over the other easier for undecided voters. This possible phenomenon explains the lack of significant results for the interaction tested for.\\

After discussing the faults of this model, alternative explanations can be examined. Perhaps these voters were able to reach choices because they lacked strong party identification. Existing literature acknowledges that undecided voters are highly susceptible to campaign forces due to the lack of their party identification. Whether or not these forces accomplish their intended goal, influencing candidate preference, has yet to be clearly uncovered. It is evident that the nuances and unique identity of undecided voters make this group a difficult population to study. While the model applied to this study did not support the hypothesis, it assists in further understanding the behavior of this voting population. \\

These results also highlight the faults of survey research in general. Within the confines of this hypothetical election, it was difficult to gauge feelings of undecidedness within the sample population. This limitation led to us defining the undecided population as individuals who did not voice preference for a major party. Examining previous research, it is clear that undecided voters encompass a more complex population, which was impossible to capture with the simplicity of this survey design. Without real elections, it is impossible to determine which voters are truly undecided towards the candidates. Restrictions in this research process made it impossible to accurately replicate debate settings, but with more time and financial resources this project may be repeated with real election data, yielding more accurate results. \\

It is unlikely that society will do away with presidential debates in their entirety. Though literature of the past and present debate their effectiveness as a campaigning tool, it is evident that they continue to play a large role in United States political culture. Even if debates are unable to sway voters to one side over the other, they serve an integral role of connecting candidates to the electorate and provide a means for outlining policy platforms and priority.\\

While the insignificant results of this study may seem discouraging, this research adds to the literature on the topic of debates. The interaction of undecided voters and presidential debates is one that necessitates future work and research. It is entirely possible that with more time, resources, and a larger sample, this research question and design could be repeated with different results. Continued research in this field will perhaps lead to a more definitive understanding of the complexities of the undecided voter. \\



\newpage

\begin{thebibliography}{20}
\setlength{\itemindent}{-0.2in} 

\bibitem{Abrajano2005} Abrajano, Marisa A. 2005. "Who evaluates a presidential candidate by using non-policy campaign messages?." \emph{Political Research Quarterly} 58(1): 55-67.

\bibitem  {Arcuri200} Arcuri, Luciano, Luigi Castelli, Silvia Galdi, Cristina Zogmaister, and Alessandro Amadori. 2008. "Predicting the vote: Implicit attitudes as predictors of the future behavior of decided and undecided voters." \emph{Political Psychology} 29(3): 369-387.

\bibitem {Baron1994} Baron, David P. 1994. "Electoral competition with informed and uninformed voters." \emph{American Political Science Review} 88(1): 33-47.

\bibitem {Benoit1999} Benoit, William L., and Allison Harthcock. 1999. “Functions of the Great Debates: Acclaims, Attacks, and Defenses in the 1960 Presidential Debates.” \emph{Communication Monographs} 66(4): 383-391.

\bibitem {Benoit2001} Benoit, William L., and Glenn J. Hansen. 2001. “Presidential debate questions and the public agenda.” \emph{Communication Quarterly} 49(2): 130-141. 

\bibitem {Benoit2003} Benoit, William L., Glenn J. Hansen, and Rebecca M. Verser. 2003. “A meta-analysis of the effects of viewing U.S. presidential debates.” \emph{Communication Monographs} 70(4): 335–350.

\bibitem {Blais2008} Blais, André, and Andrea M. L. Perrella. 2008. “Systemic Effects of Televised Candidates’ Debates.” \emph{The International Journal of Press/Politics} 13(4): 451–464.

\bibitem {Christ1985} Christ, William G. 1985. "Voter Preference and Emotion: Using Emotional Response to Classify Decided and Undecided Voters 1." \emph{Journal of Applied Social Psychology} 15(3): 237-254.

\bibitem {Clarke1994} Clarke, Harold D., and Motoshi Suzuki. 1994. “Partisan Dealignment and the Dynamics of Independence in the American Electorate, 1953–88.” \emph{British Journal of Political Science} 24(1): 57–77. 

\bibitem{Clementson2016} Clementson, David E., Paola Pascual‐Ferrá, and Michael J. Beatty. 2016. "When does a presidential candidate seem presidential and trustworthy? Campaign messages through the lens of language expectancy theory." \emph{Presidential Studies Quarterly} 46(3): 592-617.

\bibitem{Clifford2015} Clifford, Scott, Ryan M. Jewell, and Philip D. Waggoner. 2015. "Are samples drawn from Mechanical Turk valid for research on political ideology?." \emph{Research & Politics} 2(4).

\bibitem {Dalager1996} Dalager, Jon K. 1996 "Voters, Issues, and Elections: Are the Candidates' Messages Getting Through?" \emph{The Journal of Politics} 58(2): 486-515.

\bibitem {Eveland1994} Eveland Jr, William P., Douglas M. McLeod, and Amy I. Nathanson. 1994. "Reporters vs. undecided voters: An analysis of the questions asked during the 1992 presidential debates." \emph{Communication Quarterly} 42(4): 390-406.

\bibitem {Fenwick1982} Fenwick, Ian, Frederick Wiseman, John F. Becker, and James R. Heiman. 1982. "Classifying Undecided Voters in Pre-Election Polls." \emph{The Public Opinion Quarterly} 46(3): 383-391. 

\bibitem {Geer1988} Geer, John G. 1988 “The Effects of Presidential Debates on the Electorate’s Preference for Candidates.” \emph{American Politics Quarterly} 16(4): 486–501.

\bibitem {Gelman1993} Gelman, Andrew, and Gary King. 1993. "Why are American presidential election campaign polls so variable when votes are so predictable?." \emph{British Journal of Political Science} 23(4): 409-451.

\bibitem {Groser2010} Großer, Jens, and Arthur Schram. 2010. "Public opinion polls, voter turnout, and welfare: An experimental study." \emph{American Journal of Political Science} 54(3): 700-717.

\bibitem {Hillygus2003} Hillygus, D. Sunshine, and Simon Jackman. 2003. "Voter Decision Making in Election 2000: Campaign Effects, Partisan Activation, and the Clinton Legacy." \emph{American Journal of Political Science} 47(4): 583-596. 

\bibitem {Hoblert2005} Holbert, Lance R. 2005. “Debate Viewing as Mediator and Partisan Reinforcement in the Relationship Between News Use and Vote Choice.” \emph{Journal of Communication} 55(1): 85-102.

\bibitem {Holbrook1999} Holbrook, Thomas M. 1999. "Political Learning from Presidential Debates." \emph{Political Behavior} 21(1): 67-89.

\bibitem {Jarman2005} Jarman, Jeffrey W. 2005. “Political Affiliation and Presidential Debates: A Real-Time Analysis of the Effect of the Arguments Used in the Presidential Debates.” \emph{American Behavioral Scientist} 49(2): 229–242. 

\bibitem {Jerit2008} Jerit, Jennifer. 2008. "Issue Framing and Engagement: Rhetorical Strategy in Public Policy Debates." \emph{Political Behavior} 30(1): 1-24.

\bibitem {Lanoue1989} Lanoue, David J., and Peter R. Schrott. 1989. "Voters' Reactions to Televised Presidential Debates: Measurement of the Source and Magnitude of Opinion Change." \emph{Political Psychology} 10(2): 275-285. 

\bibitem {McKinney2004} McKinney, Mitchell S., and Diana B. Carlin. 2004. “Political campaign debates.” In \emph{Handbook of political communication research}, edited by Lynda L. Lawrence Erlbaum. Mahwah: Lawrence Erlbaum Associates Publishers.

\bibitem {Middleton1962} Middleton, Russell. 1962. "National TV Debates and Presidential Voting Decisions." \emph{The Public Opinion Quarterly} 26(3): 426-429. 

\bibitem {Sigelman1984} Sigelman, Lee, and Carol K. Sigelman. 1984. "Judgments of the Carter-Reagan Debate: The Eyes of the Beholders." \emph{The Public Opinion Quarterly} 48(3): 624-628.

\bibitem {Wegenast2019} Wegenast, Tim. 2010. "Uninformed voters for sale: Electoral competition, information and interest groups in the US." \emph{Kyklos} 63(2): 271-300.

\bibitem {Williams1994} Williams, Kenneth C. 1994. "Spatial Elections with Endorsements and Uninformed Voters: Some Laboratory Experiments." \emph{Public Choice} 80(1/2): 1-8.

\section*{Appendix}

Issue statements presented to survey participants:

Energy Policy:\\
\noindent{A: “We have to guard our energy companies. The EPA are so restrictive that they are putting our energy companies out of business.” \\}
\noindent{B: “We are for the first time energy independent. We do not rely on the Middle East… I think we can be the 21st century clean energy superpower.” \\}

Economic Policy:\\
\noindent{A: “We're going to cut taxes massively. We're going to cut business taxes massively. They're going to start hiring people we're going to bring the \$2.5 trillion that’s offshore back into the country.”\\}
\noindent{B: “I want us to have the biggest jobs program since World War II. Jobs in infrastructure and advanced manufacturing. I think we can compete with high wage countries and I believe we should.”\\}

Our Role in Iraq:\\
\noindent{A: “Iran is taking over Iraq. Something they've wanted to do forever, but we've made it so easy for them. So we're now going to take Mosul and you know who is going to be the beneficiary? Iran.”\\}
\noindent{B: “I will not support putting American soldiers into Iraq as an occupying force. I don't think that is in our interest, and I don't think that would be smart to do.”\\}

Healthcare:\\
\noindent{A: “We want competition. You will have the finest health care plan there is. Go[ing] to a single player plan would be a disaster.”\\}
\noindent{B: “And I'm going to fix [the Affordable Care Act]... Premiums have gotten too high, copays, deductibles, prescription drug costs and I have laid out a series of actions that we can take to try to get those costs down.”\\}

Supreme Court Nominations:\\
\noindent{A: “I am looking for judges [who are]... Highly thought of and actually very beautifully reviewed by just about everybody. But people that will respect the Constitution of the United States. And I think that this is so important.”\\}
\noindent{B: “I want to appoint Supreme Court justices who understand the way the world really works, who have real life experience. Who have not just been in a big law firm and maybe clerked for a judge and then gotten on the bench, but maybe they tried more cases.”\\}


\end{document}